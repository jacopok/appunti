\documentclass[12pt,a4paper]{article}
\usepackage[utf8]{inputenc}
\usepackage{amsmath}
\newcommand{\norm}[1]{\left\lVert#1\right\rVert}
\usepackage{amsfonts}
\usepackage{comment}
\usepackage{nicefrac}
\usepackage[margin=3.3cm]{geometry}
\usepackage{amssymb}
\usepackage[italian]{babel}
\usepackage{accents}
\usepackage{amsthm}
\usepackage[pdftex, pdfborderstyle={/S/U/W 0}]{hyperref}
\numberwithin{equation}{section}
\usepackage{commath}
\usepackage{graphicx}
%\usepackage[extreme]{savetrees}
\usepackage{bm}
\usepackage{indentfirst}
\usepackage{nicefrac}
\setcounter{tocdepth}{4}
\usepackage{fnpct}
\usepackage{centernot}

\usepackage{cool}
\Style{DSymb={\mathrm d},DShorten=true,IntegrateDifferentialDSymb=\mathrm{d}}

\usepackage{microtype}
\usepackage{cleveref}
\usepackage{tensor}

\theoremstyle{definition}
\newtheorem{definition}{Definizione}[section]
\usepackage{mathtools}
 
\theoremstyle{remark}
\newtheorem*{remark}{Remark}

\newtheorem{theorem}{Teorema}[section]
\newtheorem{corollary}{Corollario}[theorem]
\newtheorem{lemma}[theorem]{Lemma}

\usepackage{url}
\newcommand*{\defeq}{\stackrel{\text{def}}{=}}
\author{Jacopo Tissino}
\title{Modelli probabilistici}

\begin{document}

\maketitle

\section{Basi}

\begin{definition}
\emph{Esperimento aleatorio}: osservazione su un fenomeno il cui esito non è determinabile a priori.
\end{definition}

\begin{definition}
\emph{Spazio di probabilità}: terna di 

\begin{enumerate}
\item Spazio campionario $\Omega$: l'insieme degli esiti possibili;
\item Evento: elemento di $\mathcal{P}(\Omega)$;
\item Probabilità: $\mathbb{P}: \mathcal{P}: \Omega \rightarrow [0, 1]$ tale che:
\begin{enumerate}
\item $\mathbb{P}(\Omega) = 1$;
\item $\sigma$-additività: $\forall$ successione di eventi $(A_n)_{n \in \mathbb{N}}$:
\begin{equation}
\mathbb{P}\left(\displaystyle \bigcup_{n=1}^\infty A_n \right) = \sum_{n=1}^\infty \mathbb{P} (A_n)
\end{equation}
\end{enumerate}
\end{enumerate}
\end{definition}

Le operazioni logiche fra eventi sono equivalenti a quelle insiemistiche fra sottoinsiemi di $\Omega$. Valgono i teoremi di De Morgan, anche per insiemi numerabili.

Eventi ``disgiunti'' hanno intersezione nulla. Dalla $\sigma$-additività deriva l'additività finita.

Proprietà immediate: $\forall A, B \in \mathcal{P}

\begin{enumerate}
\item 
\end{enumerate}

\tableofcontents

\end{document}
