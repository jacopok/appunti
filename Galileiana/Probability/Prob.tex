\documentclass[12pt,a4paper]{article}
\usepackage[utf8]{inputenc}
\usepackage{amsmath}
\newcommand{\norm}[1]{\left\lVert#1\right\rVert}
\usepackage{amsfonts}
\usepackage{comment}
\usepackage{nicefrac}
\usepackage[margin=3.3cm]{geometry}
\usepackage{amssymb}
\usepackage[italian]{babel}
\usepackage{accents}
\usepackage{amsthm}
\usepackage[pdftex, pdfborderstyle={/S/U/W 0}]{hyperref}
\numberwithin{equation}{section}
\usepackage{commath}
\usepackage{graphicx}
%\usepackage[extreme]{savetrees}
\usepackage{bm}
\usepackage{indentfirst}
\usepackage{nicefrac}
\setcounter{tocdepth}{4}
\usepackage{fnpct}
\usepackage{centernot}

\usepackage{cool}
\Style{DSymb={\mathrm d},DShorten=true,IntegrateDifferentialDSymb=\mathrm{d}}

\usepackage{microtype}
\usepackage{cleveref}
\usepackage{tensor}

\theoremstyle{definition}
\newtheorem{definition}{Definizione}[section]
\usepackage{mathtools}
 
\theoremstyle{remark}
\newtheorem*{remark}{Remark}

\newtheorem{theorem}{Teorema}[section]
\newtheorem{corollary}{Corollario}[theorem]
\newtheorem{lemma}[theorem]{Lemma}

\usepackage{url}
\newcommand*{\defeq}{\stackrel{\text{def}}{=}}
\author{Jacopo Tissino}
\title{Modelli probabilistici}

\begin{document}

\maketitle

\section{Basi}

\begin{definition}
\emph{Esperimento aleatorio}: osservazione su un fenomeno il cui esito non è determinabile a priori.
\end{definition}

\begin{definition}
\emph{Spazio di probabilità}: terna di 

\begin{enumerate}
\item Spazio campionario $\Omega$: l'insieme degli esiti possibili;
\item Evento: elemento di $\mathcal{P}(\Omega)$;
\item Probabilità: $\mathbb{P}: \mathcal{P}: \Omega \rightarrow [0, 1]$ tale che:
\begin{enumerate}
\item $\mathbb{P}(\Omega) = 1$;
\item $\sigma$-additività: $\forall$ successione di eventi $(A_n)_{n \in \mathbb{N}}$:
\begin{equation}
\mathbb{P}\left(\displaystyle \bigcup_{n=1}^\infty A_n \right) = \sum_{n=1}^\infty \mathbb{P} (A_n)
\end{equation}
\end{enumerate}
\end{enumerate}
\end{definition}

Le operazioni logiche fra eventi sono equivalenti a quelle insiemistiche fra sottoinsiemi di $\Omega$. Valgono i teoremi di De Morgan, anche per insiemi numerabili.

Eventi ``disgiunti'' hanno intersezione nulla. Dalla $\sigma$-additività deriva l'additività finita.

Proprietà immediate: $\forall A, B \in \mathcal{P} (\Omega)$:

\begin{enumerate}
\item $\mathbb{P}(B \smallsetminus A) = \mathbb{P}(B) - \mathbb{P}(B \cap A)$;
\item $\mathbb{P}(A \cup B) = \mathbb{P}(A) + \mathbb{P}(B) - \mathbb{P}(A \cap B) \leq \mathbb{P}(A) + \mathbb{P}(B)$.
\end{enumerate}

\begin{definition}
La funzione $p: \Omega \rightarrow [0, 1]$ è una \emph{densità di probabilità} se:

\begin{enumerate}
\item $\forall \omega \in \Omega: p(\omega) \geq 0$;
\item $\sum_{\omega \in \Omega} p(\omega) = 1$. 
\end{enumerate}
\end{definition}

\begin{theorem}
Data la funzione $p: \Omega \rightarrow [0, 1]$, la funzione $\mathbb{P}: \mathcal{P}(\Omega) \rightarrow [0, 1]$ tale che $\forall A \subseteq \Omega: \mathbb{P}(A) = \sum_{\omega \in A} p(\omega)$ è una probabilità.

Vale anche il viceversa: data $\mathbb{P}: \mathcal{P}(\Omega) \rightarrow [0, 1]$ la funzione $p: \Omega \rightarrow [0, 1]$ tale che $p(\omega) = \mathbb{P} (\lbrace \omega \rbrace )$ è una densità di probabilità.
\end{theorem}

\paragraph{Esempi di spazi discreti}

Per uno spazio uniforme, per il quale $\abs{\Omega} \in \mathbb{N}$, $\forall \omega \in \Omega: p(\omega) = 1/\abs{\Omega}$.

\paragraph{Richiami di combinatoria}

Scegliamo $k$ elementi da $n$:

\begin{enumerate}
\item Disposizioni con ripetizione: $n^k$;
\item Disposizioni semplici: $\prod_{i=n-k+1}^n i = n! / (n-k+1)!$;
\item Combinazioni semplici: $C_{n, k} = { n \choose k } = n! / (k! (n-k)!)$.
\end{enumerate}

\subparagraph{Coefficienti multinomiali}

Vogliamo dividere $n$ elementi in $k$ gruppi, di cardinalità $(n_i)$, con $\sum_i n_i = n$. Si può fare in un numero di modi pari a:

\begin{equation}
{n \choose n_1, n_2, \dots, n_k } = \frac{n!}{\prod_{i=1}^k n_i !}
\end{equation}

\section{Modello di Ising}

Dato lo spazio finito $\Omega$ e data la funzione Hamiltoniana $H: \Omega \rightarrow \mathbb{R}$ (~energia) con parametro $\beta \geq 0$. Definiamo $\forall \omega \in \Omega$ la \emph{Misura di Gibbs}:

\begin{equation}
p_\beta = \frac{e^{-\beta H(\omega)}}{z_\beta} = \frac{e^{-\beta H(\omega)}}{\displaystyle\sum_{\omega \in \Omega} e^{-\beta H(\omega)}}
\end{equation}

è una ddp. Interpretazione fisica: $\beta^{-1} = k_B T$. La misura dà la probabilità di osservare un certo stato all'equilibrio.

Casi limite: $\beta \approx 0$ densità uniforme, $\beta \rightarrow \infty$ densità zero ovunque, uniforme nel minimo.

Un grafo è un insieme di vertici e spigoli: $G = (V, E), E \subseteq V\times V$. Definiamo quindi lo spazio:

\begin{equation}
\mathbb{Z} ^n = \left(\mathbb{Z} ^n, (x, y) \in \mathbb{Z} ^n:\text{d}(x, y) = 1 \right)
\end{equation}

Un sottografo finito $\Lambda \subset \mathbb{Z}^n$ è considerato con tutti i suoi spigoli: $E(\Lambda) = \lbrace (x, y) : x, y \in \Lambda, (x, y) \in E (\mathbb{Z}^n ) \rbrace$.

Ogni vertice assume valore $\pm 1$. Lo spazio è dunque $\Omega = \lbrace \pm 1 \rbrace ^{\abs{\Lambda}}$; $\sigma$ è uno stato, ovvero $\sigma \in \Omega \implies \sigma = (\sigma_k)_{k \in \Lambda}$.

Per comodità, $\text{d} (x, y) = 1 \iff x \sim y$.

Definiamo $\partial \Lambda = \lbrace x \in \Lambda : \exists y \in \Lambda^C : x\sim y \rbrace$.
Sia quindi $\tau = \lbrace \pm 1 \rbrace ^{\abs{\Lambda^C}}$ l'esterno del reticolo. Definiamo l'Hamiltoniana

\begin{equation}
H_\Lambda^\tau (\sigma ) = - \frac 12 \sum_{\substack{x, y \in \Lambda \\ x \sim y}} \sigma_x \sigma_y - \sum_{\substack{x \in \partial \Lambda \\ y \in \Lambda^C \\ x \sim y}} \sigma_x \tau_y
\end{equation}

Consideriamo solo reticoli ``quadrati'': $\Lambda_n = \lbrace -n, \dots, n \rbrace ^d \subset \mathbb{Z}^d$. Poniamo $\tau = +1$.

\tableofcontents

\end{document}
