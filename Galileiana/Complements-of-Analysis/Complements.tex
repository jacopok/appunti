\documentclass[12pt,a4paper]{report}
\usepackage[utf8]{inputenc}
\usepackage{amsmath}
\newcommand{\norm}[1]{\left\lVert#1\right\rVert}
\usepackage{amsfonts}
\usepackage{comment}
\usepackage{nicefrac}
\usepackage[margin=3.3cm]{geometry}
\usepackage{amssymb}
\usepackage{accents}
\usepackage{amsthm}
\usepackage[pdftex, pdfborderstyle={/S/U/W 0}]{hyperref}
\numberwithin{equation}{section}
\usepackage{commath}
\usepackage{graphicx}
%\usepackage[extreme]{savetrees}
\usepackage{bm}
\usepackage{indentfirst}
\usepackage{nicefrac}
\setcounter{tocdepth}{4}
\usepackage{fnpct}
\usepackage{centernot}

\usepackage{cool}
\Style{DSymb={\mathrm d},DShorten=true,IntegrateDifferentialDSymb=\mathrm{d}}

\usepackage{microtype}
\usepackage{cleveref}

\theoremstyle{definition}
\newtheorem{definition}{Definition}[section]
\usepackage{mathtools}
 
\theoremstyle{remark}
\newtheorem*{remark}{Remark}

\newtheorem{theorem}{Theorem}[section]
\newtheorem{corollary}{Corollary}[theorem]
\newtheorem{lemma}[theorem]{Lemma}

\usepackage{url}
\newcommand*{\defeq}{\stackrel{\text{def}}{=}}
\author{Jacopo Tissino}
\title{Notes on Complements of Analysis}

\begin{document}

\maketitle

\chapter{Set theory}

\section{The ZFC axioms}

\paragraph{Extensionality}

\begin{equation}
\forall x: \forall y: \forall a: x=y \iff (a \in x \iff a \in y)
\end{equation}

\paragraph{Existence of the null set}

\begin{equation}
\exists x: \forall y: y \notin x
\end{equation}

\paragraph{Foundation}

Every nonempty set contains an $\in$-minimal element:

\begin{equation}
\forall A: \exists x \in A : \forall y \in A: y \notin x
\end{equation}

This means that there cannot be an infinite $\in$ chain like $A_1 \ni A_2 \ni A_3 \ni \dots$.

We can also say that $\forall A: \exists x \in A: x \cap A = \emptyset$.

This also exclude the existence of the set of all sets: $\nexists x: \forall y: y \in x$.

\paragraph{Separation}

Given a well-defined property $P(x)$, there exists a set such that

\begin{equation}
\forall y : \forall x: x \in y \wedge P(x)
\end{equation}

This implies the existence of the empty set, and excludes Russel's paradox.

\paragraph{Pair sets}

\begin{equation}
\forall a: \forall b: \exists x: \forall y: y \in x \iff (y=a \vee y=b)
\end{equation}

This implies the existence of singlets, and of ordered pairs, defined as: $(a, b) := \lbrace \lbrace a\rbrace, \lbrace a, b\rbrace\rbrace$ ($a = \cap (a, b)$, $b = \cup (a, b) \diagdown \cap (a, b)$).

Of course, 

\begin{equation}
(a, b) = (c, d) \iff (a=c) \wedge (b=d)
\end{equation}

\paragraph{Union set axiom}

\begin{equation}
\forall x: \exists u: \forall z: \exists y: z \in u \iff (z \in y \wedge y \in x ) 
\end{equation}

The usual notation is $u = \cup x$, or $A\cup B$. This also enables us to define intersections:

\begin{equation}
A \cap B = \left\lbrace x \in  \left\lbrace A \cup B \right\rbrace : x \in A \wedge x \in B\right\rbrace
\end{equation}

\paragraph{Power set axiom}

\begin{equation}
\forall x: \exists p: \forall y: y \in p \iff y \subseteq x
\end{equation}

The usual notation is: $p = \mathcal{P}(x)$.

\paragraph{Infinity}

\begin{equation}
\exists x: \forall y: \emptyset \in x \wedge (y \in x \implies y \cup \lbrace y \rbrace \in x )
\end{equation}

\paragraph{Replacement}

\section{The Von Neumann Integers}

We define $0_{VN} = \emptyset$, and $S(n_{VN}) =n \cup \lbrace n_{VN} \rbrace$. So $1_{VN} = \lbrace \emptyset \rbrace$, $2_{VN} = \lbrace \emptyset , \lbrace \emptyset \rbrace \rbrace$, $3_{VN} = \lbrace \emptyset , \lbrace \emptyset \rbrace , \lbrace \emptyset , \lbrace \emptyset \rbrace \rbrace \rbrace$...

The $\leq$  relation is thus replaced by $\subseteq$, and $<$ by $\in$.

The axiom of infinity seems to define the VN integers, but many sets could have those properties. So, we define the property $P(x) = \emptyset \in x \wedge (y \in x \implies y \cup \lbrace y \rbrace \in x )$

We'd like to intersect all the sets satisfying $P(x)$ (HOW?)

\begin{equation}
\omega:= \lbrace k \in x : k \in Y \iff \forall Y \in \mathcal{P}(x): P(y)
\end{equation}



\tableofcontents

\end{document}
