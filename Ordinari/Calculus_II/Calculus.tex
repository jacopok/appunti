\documentclass[12pt,a4paper]{report}
\usepackage[utf8]{inputenc}
\usepackage{amsmath}
\newcommand{\norm}[1]{\left\lVert#1\right\rVert}
\usepackage{amsfonts}
\usepackage{comment}
\usepackage{nicefrac}
\usepackage[margin=3.3cm]{geometry}
\usepackage{amssymb}
\usepackage{accents}
\usepackage{amsthm}
\usepackage[pdftex, pdfborderstyle={/S/U/W 0}]{hyperref}
\numberwithin{equation}{section}
\usepackage{commath}
\usepackage{graphicx}
%\usepackage[extreme]{savetrees}
\usepackage{bm}
\usepackage{indentfirst}
\usepackage{nicefrac}
\setcounter{tocdepth}{4}
\usepackage{fnpct}
\usepackage{centernot}
\usepackage[italian]{babel}

\usepackage{cool}
\Style{DSymb={\mathrm d},DShorten=true,IntegrateDifferentialDSymb=\mathrm{d}}

\DeclareMathOperator{\sgn}{sgn}
\usepackage{microtype}
\usepackage{cleveref}

\theoremstyle{definition}
\newtheorem{definition}{Definition}[section]
\usepackage{mathtools}

\DeclarePairedDelimiter\ceil{\lceil}{\rceil}
\DeclarePairedDelimiter\floor{\lfloor}{\rfloor}

 
\theoremstyle{remark}
\newtheorem*{remark}{Remark}

\newtheorem{theorem}{Teorema}[section]
\newtheorem{corollary}{Corollario}[theorem]
\newtheorem{lemma}[theorem]{Lemma}

\usepackage{url}
\newcommand*{\defeq}{\stackrel{\text{def}}{=}}
\author{Jacopo Tissino}
\title{Appunti di Analisi II}

\begin{document}

\maketitle

\chapter{Serie}

Definite in $\mathbb{C}$. Se omettiamo gli estremi della somma, s'intende da un qualche naturale, generalmente 1, a $+\infty$.

\section{Definizioni}

Se la successione delle somme parziali ha limite finito o infinito, allora scriviamo:

\begin{equation}
\sum_{n=1}^\infty a_n = \lim_{n \rightarrow \infty} \sum_{k=1}^\infty a_k
\end{equation}

Data una serie $\sum_{n=1}^\infty a_n$, $(a_n)$ è la sua successione dei termini generali.

\begin{theorem}
Se la serie $\sum_{n=1}^\infty a_n$ converge, allora $\lim_{n \rightarrow \infty} a_n = 0$
\end{theorem}

\section{Serie notevoli}

\paragraph{Geometrica} Se $z \in \mathbb{C}$, $\abs{z} < 1$, allora

\begin{equation}
\sum_{n=1}^\infty z^n = \frac{1}{1-z}
\end{equation}

\paragraph{Telescopica} Data $(a_n)_{n \in \mathbb{N}}$ convergente a $\ell$, se $(b_n)_{n \in \mathbb{N}}$ è tale che $b_n = a_{n+1} - a_n$, allora

\begin{equation}
\sum_{n=0}^\infty b_n = \ell - a_0
\end{equation}

\paragraph{Armonica generalizzata}

Per $ p \leq 1$, la serie

\begin{equation}
\sum_{n=1}^\infty \frac{1}{n^p}
\end{equation}

diverge, converge invece per $p>1$.

\section{Criteri}

\paragraph{Confronto} Date le serie $\sum a_n$ e $\sum b_n$, entrambe a termini positivi, se $a_n \leq b_n$ definitivamente

\begin{enumerate}
\item se $\sum a_n$ diverge allora $\sum b_n$ diverge;
\item se $\sum b_n$ converge allora $\sum a_n$ converge;
\end{enumerate}

\paragraph{Confronto asintotico} Date le serie $\sum a_n$ e $\sum b_n$, entrambe a termini positivi, se

\begin{equation}
\lim_{n \rightarrow \infty} \frac{a_n}{b_n} = \ell \in \bar{\mathbb{R}}^+
\end{equation}

allora

\begin{enumerate}
\item se $\ell \in (0, +\infty )$, le serie hanno lo stesso carattere;
\item se $\ell = 0$, $a_n$ diverge $\implies b_n$ diverge e $b_n$ converge $\implies a_n$ converge;
\item se $\ell = +\infty$, $b_n$ diverge $\implies a_n$ diverge e $a_n$ converge $\implies b_n$ converge.
\end{enumerate}

\paragraph{Rapporto}

Data $\sum a_n$ a termini positivi, se $\exists h \in (0, 1)$ tale che 

\begin{equation}
\frac{a_{n+1}}{a_n} \leq h
\end{equation}

definitivamente, allora $\sum a_n$ converge. Se $\exists h > 1$ tale che per infiniti valori di $n$

\begin{equation}
\frac{a_{n+1}}{a_n} > h
\end{equation}

allora $\sum a_n$ diverge.

\paragraph{Rapporto asintotico}

Data $\sum a_n$ a termini positivi, e $\ell = \limsup_{n \rightarrow \infty} a_{n+1}/a_n.$ Allora $\ell \in [0, +\infty]$. Se $\ell < 1$, $\sum a_n$ converge. Se $\ell > 1$, $\sum a_n$ diverge.

\paragraph{Radice ($n$-esima)}

Data $\sum a_n$ a termini positivi, se $\exists h \in (0, 1)$ tale che 

\begin{equation}
\sqrt[n]{a_n} \leq h
\end{equation}

definitivamente, allora $\sum a_n$ converge. Se $\exists h > 1$ tale che per infiniti valori di $n$

\begin{equation}
\sqrt[n]{a_n} > h
\end{equation}

allora $\sum a_n$ diverge.

\paragraph{Radice asintotica}

Data $\sum a_n$ a termini positivi, e $\ell = \limsup_{n \rightarrow \infty} \sqrt[n]{a_n}.$ Allora $\ell \in [0, +\infty]$. Se $\ell < 1$, $\sum a_n$ converge. Se $\ell > 1$, $\sum a_n$ diverge.

\begin{theorem}
(Indimostrato) Data una successione $(a_n)_{n \in \mathbb{N}}$ a termini positivi:

\begin{equation}
\liminf_{n \rightarrow \infty} \frac{a_{n+1}}{a_n} \leq \liminf_{n \rightarrow \infty} \sqrt[n]{a_n}
\leq \limsup_{n \rightarrow \infty} \sqrt[n]{a_n} \leq \limsup_{n \rightarrow \infty} \frac{a_{n+1}}{a_n}
\end{equation}
\end{theorem}

\paragraph{Condensazione di Cauchy}

Data $\sum a_n$ a termini positivi, con termine generale decrescente, le serie $\sum a_n$ e $\sum 2^n a_{2^n}$ hanno lo stesso carattere.

\paragraph{Leibniz}

Data la successione $a_n$ a termini positivi infinitesima con $a_{n+1} \leq a_n$, la serie 

\begin{equation}
\sum b_n = \sum (-1)^n a_n
\end{equation}

converge, e

\begin{equation}
\abs{\sum_{n=k}^\infty b_n} \leq a_{k+1}
\end{equation}

\paragraph{Convergenza assoluta}

Una serie $\sum z_n$ si dice assolutamente convergente se converge la serie $\sum \abs{z_n}$. Se una serie è assolutamente convergente, allora è convergente, e vale $\abs{\sum z_n} \leq \sum \abs{z_n}$.

\section{Riordinamenti}

Data $\sum a_n$ a termini reali, e una biiezione $\sigma: \mathbb{N} \rightarrow \mathbb{N}$, definifiamo \emph{riordinamento} la serie $\sum a_{\sigma (n)}$.

\begin{itemize}
\item Se $\sum a_n$ è assolutamente convergente, $\sum a_n = \sum a_{\sigma (n)}$;
\item se $\sum a_n$ è convergente ma assolutamente divergente, $\forall L \in \bar{\mathbb{R}}: \exists \sigma : \mathbb{N} \rightarrow \mathbb{N}: \sum a_{\sigma (n)} = L$.
\end{itemize}

\chapter{Integrali generalizzati}

Una funzione $f: [a, b) \rightarrow \mathbb{R}$, $\mathbb{R}$-integrabile in $[a, c]$ $\forall c \in [a, b)$, è integrabile in senso generalizzato se:

\begin{equation}
\lim_{c \rightarrow b^-} \int^{c}_{a} f(x) \, \text{d} x \in \mathbb{R}
\end{equation}

$f$ è integrabile in senso generalizzato in $(a, b)$ se lo è in $(a, k]$ e in $[k, b)$ per ogni $k \in (a, b)$. 

\section{Criteri di convergenza}

\paragraph{Esistenza del limite}

Se $f: [a, b) \rightarrow \mathbb{R}$, $b \in \bar{\mathbb{R}}$, $f \geq 0$,

\begin{equation}
\exists \lim_{c \rightarrow b^-} \int^{c}_{a} f(x) \, \text{d} x
\end{equation}

\paragraph{Confronto}

Se $f$ e $g$ sono $\mathbb{R}$-integrabili in $(a, c] \cup [c, b)$ $\forall c \in (a, b)$, allora se $0\leq f \leq g$, se $g$ è integrabile in senso generalizzato lo è anche $f$, e se $f$ non lo è non lo è neanche $g$.

\paragraph{Confronto asintotico}

Date $f, g: [a, b) \rightarrow \mathbb{R}$, $b \in \bar{\mathbb{R}}$, $\mathbb{R}$-integrabili in $(a, c] \cup [c, b)$ $\forall c \in (a, b)$, allora se $\exists \lim_{x\rightarrow b^{-}} f/g = \ell \in [0, +\infty]$:

\begin{enumerate}
\item se $\ell \in (0, +\infty)$ i due integrali hanno lo stesso carattere;
\item se $\ell = 0$, $g$ converge $\implies f$ converge, $f$ diverge $\implies g$ diverge;
\item se $\ell = +\infty$, $f$ converge $\implies g$ converge, $g$ diverge $\implies f$ diverge;
\end{enumerate}

\paragraph{Convergenza assoluta}

Data $f: [a, +\infty) \rightarrow \mathbb{R}$, se $\abs{f}$ è integrabile in senso generalizzato allora lo è anche $f$, e 

\begin{equation}
\abs{\int^{+\infty}_{a} f (x) \, \text{d} x} \leq \int^{+\infty}_{a} \abs{f(x)} \, \text{d}x
\end{equation}

\paragraph{Serie e integrali}

Se $f: [0, +\infty ) \rightarrow \mathbb{R}$ positiva e decrescente, allora

\begin{equation}
\int^{+\infty}_{0} f (x) \, \text{d} x \,\,\text{ e }\, \sum_{n=0}^\infty f(n) \text{ hanno lo stesso carattere.}
\end{equation}

Chiaramente, basta che la funzione sia definitivamente decrescente.

\section{Integrali notevoli}

Se $P(x) = x^2 -2px +q $ ha una o due radici reali, $\int_\mathbb{R} (P(x))^{-1} \, \text{d} x$ diverge. Altrimenti, ovvero se $q-p^2 > 0$, 

\begin{equation}
\int_{\mathbb{R}} \frac{1}{x^2 - 2px +q} \, \text{d}x = \frac{\pi}{\sqrt{q-p^2}}
\end{equation}
 
\chapter{Successioni e serie di funzioni}



\chapter{Calcolo differenziale multivariato}

\chapter{Curve e 1-forme differenziali in $\mathbb{R}^n$}

\chapter{Invertibilità locale e funzione implicita}

\tableofcontents

\end{document}
