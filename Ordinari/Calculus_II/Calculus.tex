\documentclass[12pt,a4paper]{report}
\usepackage[utf8]{inputenc}
\usepackage{amsmath}
\newcommand{\norm}[1]{\left\lVert#1\right\rVert}
\usepackage{amsfonts}
\usepackage{comment}
\usepackage{nicefrac}
\usepackage[margin=3.3cm]{geometry}
\usepackage{amssymb}
\usepackage{accents}
\usepackage{amsthm}
\usepackage[pdftex, pdfborderstyle={/S/U/W 0}]{hyperref}
\numberwithin{equation}{section}
\usepackage{commath}
\usepackage{graphicx}
%\usepackage[extreme]{savetrees}
\usepackage{bm}
\usepackage{indentfirst}
\usepackage{nicefrac}
\setcounter{tocdepth}{4}
\usepackage{fnpct}
\usepackage{centernot}
\usepackage[italian]{babel}

\usepackage{cool}
\Style{DSymb={\mathrm d},DShorten=true,IntegrateDifferentialDSymb=\mathrm{d}}

\DeclareMathOperator{\sgn}{sgn}
\usepackage{microtype}
\usepackage{cleveref}

\theoremstyle{definition}
\newtheorem{definition}{Definition}[section]
\usepackage{mathtools}

\DeclarePairedDelimiter\ceil{\lceil}{\rceil}
\DeclarePairedDelimiter\floor{\lfloor}{\rfloor}

 
\theoremstyle{remark}
\newtheorem*{remark}{Remark}

\newtheorem{theorem}{Teorema}[section]
\newtheorem{corollary}{Corollario}[theorem]
\newtheorem{lemma}[theorem]{Lemma}

\usepackage{url}
\newcommand*{\defeq}{\stackrel{\text{def}}{=}}
\author{Jacopo Tissino}
\title{Appunti di Analisi II}

\begin{document}

\maketitle

\chapter{Serie}

Definite in $\mathbb{C}$. Se omettiamo gli estremi della somma, s'intende da un qualche naturale, generalmente 1, a $+\infty$.

\section{Definizioni}

Se la successione delle somme parziali ha limite finito o infinito, allora scriviamo:

\begin{equation}
\sum_{n=1}^\infty a_n = \lim_{n \rightarrow \infty} \sum_{k=1}^\infty a_k
\end{equation}

Data una serie $\sum_{n=1}^\infty a_n$, $(a_n)$ è la sua successione dei termini generali.

\begin{theorem}
Se la serie $\sum_{n=1}^\infty a_n$ converge, allora $\lim_{n \rightarrow \infty} a_n = 0$
\end{theorem}

\section{Serie notevoli}

\paragraph{Geometrica} Se $z \in \mathbb{C}$, $\abs{z} < 1$, allora

\begin{equation}
\sum_{n=1}^\infty z^n = \frac{1}{1-z}
\end{equation}

\paragraph{Telescopica} Data $(a_n)_{n \in \mathbb{N}}$ convergente a $\ell$, se $(b_n)_{n \in \mathbb{N}}$ è tale che $b_n = a_{n+1} - a_n$, allora

\begin{equation}
\sum_{n=0}^\infty b_n = \ell - a_0
\end{equation}

\paragraph{Armonica generalizzata}

Per $ p \leq 1$, la serie

\begin{equation}
\sum_{n=1}^\infty \frac{1}{n^p}
\end{equation}

diverge, converge invece per $p>1$.

\section{Criteri}

\paragraph{Confronto} Date le serie $\sum a_n$ e $\sum b_n$, entrambe a termini positivi, se $a_n \leq b_n$ definitivamente

\begin{enumerate}
\item se $\sum a_n$ diverge allora $\sum b_n$ diverge;
\item se $\sum b_n$ converge allora $\sum a_n$ converge;
\end{enumerate}

\paragraph{Confronto asintotico} Date le serie $\sum a_n$ e $\sum b_n$, entrambe a termini positivi, se

\begin{equation}
\lim_{n \rightarrow \infty} \frac{a_n}{b_n} = \ell \in \bar{\mathbb{R}}^+
\end{equation}

allora

\begin{enumerate}
\item se $\ell \in (0, +\infty )$, le serie hanno lo stesso carattere;
\item se $\ell = 0$, $a_n$ diverge $\implies b_n$ diverge e $b_n$ converge $\implies a_n$ converge;
\item se $\ell = +\infty$, $b_n$ diverge $\implies a_n$ diverge e $a_n$ converge $\implies b_n$ converge.
\end{enumerate}

\paragraph{Rapporto}

Data $\sum a_n$ a termini positivi, se $\exists h \in (0, 1)$ tale che 

\begin{equation}
\frac{a_{n+1}}{a_n} \leq h
\end{equation}

definitivamente, allora $\sum a_n$ converge. Se $\exists h > 1$ tale che per infiniti valori di $n$

\begin{equation}
\frac{a_{n+1}}{a_n} > h
\end{equation}

allora $\sum a_n$ diverge.

\paragraph{Rapporto asintotico}

Data $\sum a_n$ a termini positivi, e $\ell = \limsup_{n \rightarrow \infty} a_{n+1}/a_n.$ Allora $\ell \in [0, +\infty]$. Se $\ell < 1$, $\sum a_n$ converge. Se $\ell > 1$, $\sum a_n$ diverge.

\paragraph{Radice ($n$-esima)}

Data $\sum a_n$ a termini positivi, se $\exists h \in (0, 1)$ tale che 

\begin{equation}
\sqrt[n]{a_n} \leq h
\end{equation}

definitivamente, allora $\sum a_n$ converge. Se $\exists h > 1$ tale che per infiniti valori di $n$

\begin{equation}
\sqrt[n]{a_n} > h
\end{equation}

allora $\sum a_n$ diverge.

\paragraph{Radice asintotica}

Data $\sum a_n$ a termini positivi, e $\ell = \limsup_{n \rightarrow \infty} \sqrt[n]{a_n}.$ Allora $\ell \in [0, +\infty]$. Se $\ell < 1$, $\sum a_n$ converge. Se $\ell > 1$, $\sum a_n$ diverge.

\begin{theorem}
(Indimostrato) Data una successione $(a_n)_{n \in \mathbb{N}}$ a termini positivi:

\begin{equation}
\liminf_{n \rightarrow \infty} \frac{a_{n+1}}{a_n} \leq \liminf_{n \rightarrow \infty} \sqrt[n]{a_n}
\leq \limsup_{n \rightarrow \infty} \sqrt[n]{a_n} \leq \limsup_{n \rightarrow \infty} \frac{a_{n+1}}{a_n}
\end{equation}
\end{theorem}

\paragraph{Condensazione di Cauchy}

Data $\sum a_n$ a termini positivi, con termine generale decrescente, le serie $\sum a_n$ e $\sum 2^n a_{2^n}$ hanno lo stesso carattere.

\paragraph{Leibniz}

Data la successione $a_n$ a termini positivi infinitesima con $a_{n+1} \leq a_n$, la serie 

\begin{equation}
\sum b_n = \sum (-1)^n a_n
\end{equation}

converge, e

\begin{equation}
\abs{\sum_{n=k}^\infty b_n} \leq a_{k+1}
\end{equation}

\paragraph{Convergenza assoluta}

Una serie $\sum z_n$ si dice assolutamente convergente se converge la serie $\sum \abs{z_n}$. Se una serie è assolutamente convergente, allora è convergente, e vale $\abs{\sum z_n} \leq \sum \abs{z_n}$.

\chapter{Integrali generalizzati}

\chapter{Spazi metrici}

\chapter{Serie di funzioni}

\chapter{Calcolo differenziale multivariato}

\chapter{Curve e 1-forme differenziali in $\mathbb{R}^n$}

\chapter{Invertibilità locale e funzione implicita}

\tableofcontents

\end{document}
