\documentclass[12pt,a4paper]{report}
\usepackage[utf8]{inputenc}
\usepackage{amsmath}
\newcommand{\norm}[1]{\left\lVert#1\right\rVert}
\usepackage{amsfonts}
\usepackage{comment}
\usepackage{nicefrac}
\usepackage[margin=3.3cm]{geometry}
\usepackage{amssymb}
\usepackage{accents}
\usepackage{amsthm}
\usepackage[pdftex, pdfborderstyle={/S/U/W 0}]{hyperref}
\numberwithin{equation}{section}
\usepackage{commath}
\usepackage{graphicx}
%\usepackage[extreme]{savetrees}
\usepackage{bm}
\usepackage{indentfirst}
\usepackage{nicefrac}
\setcounter{tocdepth}{4}
\usepackage{fnpct}
\usepackage{centernot}

\usepackage{cool}
\Style{DSymb={\mathrm d},DShorten=true,IntegrateDifferentialDSymb=\mathrm{d}}

\usepackage{microtype}
\usepackage{cleveref}

\theoremstyle{definition}
\newtheorem{definition}{Definition}[section]
\usepackage{mathtools}
 
\theoremstyle{remark}
\newtheorem*{remark}{Remark}

\newtheorem{theorem}{Theorem}[section]
\newtheorem{corollary}{Corollary}[theorem]
\newtheorem{lemma}[theorem]{Lemma}

%\usepackage{mxchem}
\usepackage{url}
\newcommand*{\defeq}{\stackrel{\text{def}}{=}}
\author{Jacopo Tissino}
\title{Chimica}

\begin{document}

\maketitle

\chapter{Introduction}

Chemistry studies the properties, the structure and the transformations of matter: it is a global science.

The structure comes from the ``lattice'': example, graphite vs. diamond.

Physical states of matter (everything which has mass and occupies space):

\begin{itemize}
\item Solid: fixed shape and volume
\item Liquid: variable shape, fixed volume
\item Gas: variable shape and volume
\end{itemize}

Matter: (smartlist)
    Mixes (can be separated)
        Homogeneous (ex: separation of methanol and ethanol)
        Heterogeneous
    Pure substances
        Elements
            Molecular
            Atomical
        (Composti)
            Molecular
            Ionic

Physical transformation: only the state or the look of the substance are changed.
Physical properties: colour, odour, all that changes without the composition
Chemical transformation: the composition is also changed.
Chemical properties: they change when the composition changes.

\paragraph{Atomic theory} Democritus, Plato and Aristotle, then a very late development (Mostly after 1800).

Of course, we use the scientific method.

\paragraph{Law of Conservation of Mass} During a chemical reaction, matter cannot be destroyed nor created: it only changes form. So the total mass of the reactants is equal to the mass of the product.

\tableofcontents

\end{document}
