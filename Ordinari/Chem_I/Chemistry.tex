\documentclass[12pt,a4paper]{report}
\usepackage[utf8]{inputenc}
\usepackage{amsmath}
\newcommand{\norm}[1]{\left\lVert#1\right\rVert}
\usepackage{amsfonts}
\usepackage{comment}
\usepackage{nicefrac}
\usepackage[margin=3.3cm]{geometry}
\usepackage{amssymb}
\usepackage{accents}
\usepackage{amsthm}
\usepackage[pdftex, pdfborderstyle={/S/U/W 0}]{hyperref}
\numberwithin{equation}{section}
\usepackage{commath}
\usepackage{graphicx}
%\usepackage[extreme]{savetrees}
\usepackage{bm}
\usepackage{indentfirst}
\usepackage{nicefrac}
\setcounter{tocdepth}{4}
\usepackage{fnpct}
\usepackage{centernot}
\usepackage{chemfig}
\usepackage[version=3]{mhchem}

\usepackage{cool}
\Style{DSymb={\mathrm d},DShorten=true,IntegrateDifferentialDSymb=\mathrm{d}}

\usepackage{microtype}
\usepackage{cleveref}

\theoremstyle{definition}
\newtheorem{definition}{Definition}[section]
\usepackage{mathtools}
 
\theoremstyle{remark}
\newtheorem*{remark}{Remark}

\newtheorem{theorem}{Theorem}[section]
\newtheorem{corollary}{Corollary}[theorem]
\newtheorem{lemma}[theorem]{Lemma}

%\usepackage{mxchem}
\usepackage{siunitx}
\usepackage{url}
\newcommand*{\defeq}{\stackrel{\text{def}}{=}}
\author{Jacopo Tissino}
\title{Chimica}

\begin{document}

\maketitle

\chapter{Introduction}

\section{History of chemistry \& basics}

Chemistry studies the properties, the structure and the transformations of matter: it is a global science.

The structure comes from the ``lattice'': example, graphite vs. diamond.

Physical states of matter (everything which has mass and occupies space):

\begin{itemize}
\item Solid: fixed shape and volume
\item Liquid: variable shape, fixed volume
\item Gas: variable shape and volume
\end{itemize}

Matter: (smartlist)
    Mixes (can be separated)
        Homogeneous (ex: separation of methanol and ethanol)
        Heterogeneous
    Pure substances
        Elements
            Molecular
            Atomical
        (Composti)
            Molecular
            Ionic

Physical transformation: only the state or the look of the substance are changed.
Physical properties: colour, odour, all that changes without the composition
Chemical transformation: the composition is also changed.
Chemical properties: they change when the composition changes.

\paragraph{Atomic theory} Democritus, Plato and Aristotle, then a very late development (Mostly after 1800).

Of course, we use the scientific method.

\paragraph{Law of Conservation of Mass} During a chemical reaction, matter cannot be destroyed nor created: it only changes form. So the total mass of the reactants is equal to the mass of the product.

\paragraph{Proust's Law} Definite proportions

\paragraph{Dalton's Law} The masses of a compound when it reacts with another to form different compounds have between then small rational ratios.

\paragraph{Atomic Theory} The atomic number is the number of protons in an atom, denoted $Z$; the mass number is the sum of the protons and the neutrons, denoted $A$. A nuclide of the element $E$ is denoted $\ce{_Z^AE}$ or $\ce{E-A}$.

\section{Measuring mass}

Using kilograms is impractical, so we define the ``uam'' as equal to one twelveth of the mass of a $\ce{^{12}C}$.

Atomic mass is the weighted average of the isotopic masses.

The mass of a nucleus is usually less than the sum of the masses of the particles; this comes from the fact that $E = m c^2$.

\subparagraph{Moles}

A mole is defined as $N_A$ of something, where $N_A$ is the number of atoms in 12 grams of $\ce{^{12}C}$.

\begin{equation}
N_A \approx \SI{6.0221415E23}{}
\end{equation}

\subparagraph{Molar mass}

It is the mass (in grams) of a mole of something, expressed in $\SI{}{g/mol}$. It is numerically equal to the atomic mass.
So

\begin{equation}
MM = \frac{g}{n}
\end{equation}

\paragraph{Metals}

Good conductors. They are usually oxidated when reacting with non-metals. They are usually solid at STP.

\paragraph{Metalloids}

Semiconductors. They act as (non-)metals when reacting with non-(metals).

\paragraph{Non-metals}

Usually di-atomic. Poor conductors. Exist in all states. The are usually reduced.

\paragraph{Divisions of the periodic table}

Groups are the columns in the periodic table, periods are rows.

\chapter{Molecules \& compounds}

\section{Pure substances}

We divide pure substances into elements (atomic or molecular) or compounds (molecular or ionic).

Molecular elements are usually diatomic, but sometimes (like in $\ce{S8}$) there are more atoms.

\paragraph{Types of formulas}

\begin{itemize}
\item Minimal (brute formula): just the elements and the numerical ratios;
\item Molecular: the elements and the exact number of each atom. To get it we need the molecular mass of the compound;
\item Structural: the elements, their number, how they are placed in space and their bonds;
\item Molecular models: spheres and sticks, or ``full'' molecular models where the atoms fill the spaces;
\item Sterical formula: from VSEPR theory, VB and OM.
\end{itemize}

\section{Ionic compounds} They are formed by an anion (usually a non-metal) and a cation (usually a metal). We need a ``unit formula'', with the types of ions and their ratios. Ions are atoms with nonzero electric charge.

To form cations or anions we have reactions like:

\begin{align}
\ce{X &-> X^{n+} + n \mathit{e}-}\\
\ce{X + n \mathit{e}- &-> X^{n-}}
\end{align}

Atoms in the 1A-3A groups form ions with charge equal to their group number (their extra electrons).
Non-metals usually form ions with charge equal to eight minus their group number (the electrons they are missing).

There are also polyatomic ions (like \ce{CO3^{2-}}).

To do: learn all the polyatomic ions. They are all negative, except for \ce{NH4^+}.

When an ionic compound forms, its total charge is always 0: this is useful when balancing reactions and finding formulas.

We always write the cation first and the anion last. We should only speak of the ``formula mass'' of an ionic compound.

\tableofcontents

\end{document}
