\documentclass[12pt,a4paper]{report}
\usepackage[utf8]{inputenc}
\usepackage{amsmath}
\newcommand{\norm}[1]{\left\lVert#1\right\rVert}
\usepackage{amsfonts}
\usepackage{comment}
\usepackage{nicefrac}
\usepackage[margin=3.3cm]{geometry}
\usepackage{amssymb}
\usepackage{accents}
\usepackage{amsthm}
\usepackage[pdftex, pdfborderstyle={/S/U/W 0}]{hyperref}
\numberwithin{equation}{section}
\usepackage{commath}
\usepackage{graphicx}
%\usepackage[extreme]{savetrees}
\usepackage{bm}
\usepackage{indentfirst}
\usepackage{nicefrac}
\setcounter{tocdepth}{4}
\usepackage{fnpct}
\usepackage{centernot}

\usepackage{cool}
\Style{DSymb={\mathrm d},DShorten=true,IntegrateDifferentialDSymb=\mathrm{d}}

\usepackage{microtype}
\usepackage{cleveref}

\theoremstyle{definition}
\newtheorem{definition}{Definition}[section]
\usepackage{mathtools}
 
\theoremstyle{remark}
\newtheorem*{remark}{Remark}

\newtheorem{theorem}{Theorem}[section]
\newtheorem{corollary}{Corollary}[theorem]
\newtheorem{lemma}[theorem]{Lemma}

\usepackage{url}
\newcommand*{\defeq}{\stackrel{\text{def}}{=}}
\author{Jacopo Tissino}
\title{Notes on Calculus I}

\begin{document}

\maketitle

\chapter{Naïve set theory}

\section{Basic sets}

\begin{equation}
\mathbb{N} \subseteq \mathbb{Z} \subseteq \mathbb{Q} \subseteq \mathbb{R}, \qquad 0 \not \in \mathbb{N}
\end{equation}

\begin{equation}
\mathbb{R}^+ := \lbrace x \in \mathbb{R} : x>0 \rbrace
\end{equation}

\paragraph{Remarks}

\begin{itemize}
\item ``$\in$'' is for elements belonging to sets, ``$\subseteq$'' is for subsets
\item $\lbrace x \rbrace \neq x$: the first is a set with $x$ as its only element, and is called a ``singlet''
\item $\subsetneq$  means ``is a subset of, but not equal to''
\item the elements of $\mathcal{P}(A)$ are precisely all the subsets of $A$
\item $\sharp A$ is the cardinality of $A$
\item $\sharp \mathcal{P} (A) = 2^{\sharp A}$
\end{itemize}

The naïve definitions of $A \cup B$, $A \cap B$, $A \smallsetminus B$ are given.

\paragraph{Properties}

\begin{itemize}
\item $A = (A \cap B ) \cup (A \smallsetminus B)$
\item $(A \cap B) \cap (A \smallsetminus B) = \emptyset$
\item $C \cap (A \cup B) = (C \cap A) \cup (C \cap B)$
\item $C \cup (A \cap B) = (C \cup A) \cap (C \cup B)$
\end{itemize}

\paragraph{Complement}

\begin{definition}
With respect to a ``universe'' set $U$, we define the complement of $A$ as $U \smallsetminus A$, denoted $A^C$.
\end{definition}

The following hold:

\begin{itemize}
\item $(A \cup B)^C = A^C \cap B^C$
\item $(A \cap B)^C = A^C \cup B^C$
\end{itemize}

\paragraph{Cartesian product}

\begin{definition}
An \emph{ordered pair} is a set of the form $\lbrace \lbrace x \rbrace , \lbrace x, y \rbrace \rbrace$, denoted $(x, y)$ (where order matters).
\end{definition}

\begin{definition}
We define the \emph{cartesian product} $A\times B$ of two sets $A$ and $B$ as:

\begin{equation}
A\times B := \left\lbrace (a, b) : a \in A, b \in B \right\rbrace
\end{equation}
\end{definition}

\section{Propositional logic}

\paragraph{Implication}
\begin{definition}
\begin{equation}
p\implies q \iff (\neg p ) \vee q
\end{equation}
\end{definition}

\paragraph{Double implication}
\begin{definition}
\begin{equation}
(p \iff q) \iff (p \implies q \wedge q \implies p )
\end{equation}
\end{definition}

\paragraph{Quantifiers}

$P(x)$ is a \emph{predicate}. We say that $\forall x : P(x)$ if $P(x)$ is true independently of $x$, and that

\begin{equation}
\exists x : P(x) \iff \neg ( \forall x : \neg P(x) )
\end{equation}

\chapter{Number sets}

\section{The set $\mathbb{N}$}

\paragraph{The Peano axioms}

\begin{enumerate}
\item $1 \in \mathbb{N}$
\item $\forall n \in \mathbb{N}:  \exists S (n) : \mathbb{N} \rightarrow \mathbb{N}$
\item $\forall n \in \mathbb{N}: S(n) \neq 1$
\item $\forall m, n \in \mathbb{N}: m \neq n \implies S(n) \neq S(m)$ or $S(n) = S(m) \implies n = m$
\item $(A \subseteq \mathbb{N} )\wedge (1 \in A )\wedge (n  \in A \implies S(n) \in A ) \implies A = \mathbb{N}$
\end{enumerate}

Any set that verifies these axioms is isomorphic to $\mathbb{N}$. $\mathbb{R}^{+}$, for example only satisfies the first 4.

\section{Induction}

If $P(n)$ is a proposition, $P(1)$ and $P(n) \implies P(n+1)$\footnote{We introduce the notation $n+1$ to signify $S(n)$.} (the \emph{inductive hypothesis}); then $\forall n \in \mathbb{N}, P(n)$.

\begin{proof}
Define $A := \lbrace n \in \mathbb{N}: P(n) \rbrace$. By axiom 5, $A = \mathbb{N}$.
\end{proof}

If a property $a': P(k)$ holds for some $k \in \mathbb{N}$ and $b': P(n) \implies P(n+1)$, then $\forall n \geq k : P(n)$.

\paragraph{Examples}

Example: we can show by induction that

\begin{equation}
\sum_{i=1}^n = \frac{n (n+1)}{2}
\end{equation}

Example 2: we can show by induction that $P(\sharp A):\sharp \mathcal{P}(A) = 2^{\sharp A}$. 

\begin{proof}
$P(1)$ is true. 
We see that for any $A$ we can take an element such that $A = \lbrace a \rbrace \cup B$. Then for any subset $I$, either $a \in I$ or $a \not \in I$. If $a \in I$, $I = \lbrace a \rbrace \cup J$, but there are $2^n$ possible $J$s. If $a \not \in I$, we have $2^n$ $I'$s. So there are $2^{n+1}$ possible subsets.
\end{proof}

Example 3: show that $n! > 2^n$, which is true for $n>3$.

\begin{proof}
We will use the second form of the induction principle. $P(4)$ is true. If $n>4$ and $n! > 2^n$, we need to show that $(n+1) n!> 2 \cdot 2^n$. But $n+1>2$ by hypothesis, so the inequality always holds.
\end{proof}

Observation: the notation $1 + 2+ 3 + 4+ \cdots + n$ is unclear, we should use $\sum_{i=1}^n i$.

Recursive formulas: we know the first term, and an algorithm to derive any term from the one before it, such as the definition of the factorial:

\begin{equation}
\begin{cases}
0! = 1\\
(n+1)! = (n+1) n!
\end{cases}
\end{equation}

Another example is the sequence:

\begin{equation}
\begin{cases}
S_1 = 2\\
S_{n+1} = S_n + (2n+1)
\end{cases}
\end{equation}


\begin{proof}[Proof that $S_n = n^2 + 1$]
Assume that $S_n = n^2 +1$. Then $S_{n+1} = n^2 + 1 + 2n + 1 = (n+1)^2 + 1$.
\end{proof}

Formal definition of summation:

\begin{equation}
\sum_{i=0}^n a_i =a_0 + a_1 + a_2 + a_3 + \cdots + a_n
\end{equation}

by recursion: for an increment in $n$, we just add the $n+1$-th term. So it comes down to the formal definition of induction.

Example: show by induction that

\begin{equation}
\forall a \in \mathbb{N} \vee a=0 \quad (a\neq 1): \quad \sum_{k=0}^n a^k = \frac{1-a^{n+1}}{1-a}
\end{equation}

See the property for $n=0$. Suppose that the property holds for $n$, show it for $n+1$:

\begin{equation}
\sum_{i=0}^{n+1} a_k = \sum_{k=0}^n a^k + a^{n+1} = \frac{1-a^{n+1}}{1-a} + a^{n+1} = \frac{1-a^{n+2}}{1-a}
\end{equation}

To do: given two real numbers, $\forall n \in \mathbb{N}$, show that

\begin{equation}
(a+b)^n = \sum_{k=0}^n {n\choose k} a^k b^{n-k}
\end{equation}

\section{Groups}

We have a set $A$ with an operation $*$: $a, b \in A \rightarrow a*b \in A$. Es: $A$ strings, $*$ concatenation. $(A, *)$ is a group if the following are satisfied:

\begin{enumerate}
\item Associativity: $\forall a, b, c \in A: \quad a* (b*c) = (a*b)*c$
\item Identity: $\exists e \in A: \forall a \in A: \quad e*a = a*e = a$
\item Inverse: $\forall a \in A: \exists a^{-1} \in A: \quad a* a^{-1} = a^{-1} * a = e$
\end{enumerate}

$(A, *)$ is also a commutative group if $\forall a, b \in A: \quad a*b=b*a$

Examples: $(\mathbb{N}, +)$ is not a group: there is no identity, but even if we add 0 there is no inverse. $(\mathbb{Z}, +)$ is a commutative group. $(\mathbb{R}, \times)$ is not a group because 0 has no inverse. $(\mathbb{R}_0, \times)$ is hower a group, as is $(\mathbb{R}_0, +)$. $(\mathbb{Z}_0, \times)$ is not a group because there is no inverse: $(\mathbb{Q}_0, \times)$ is a commutative group.

So we introduce $\mathbb{Q}$:

\begin{equation}
\mathbb{Q}:= \left\lbrace \frac{a}{b}, a \in \mathbb{Z}, b\in \mathbb{N}\right\rbrace  / \sim
\end{equation}

but $ad \sim bc$ and $\mathbb{Q}$ is isomorphic to $\mathbb{Z}$

\begin{definition}
Order relation:

\begin{equation}
\frac a b \leq \frac cd \iff ad \leq bc 
\end{equation}
\end{definition}

\begin{definition}
Sum and product, subtraction and division are analogous:

\begin{equation}
\frac ab + \frac cd = \frac{ad+bc}{bd}
\end{equation}
\begin{equation}
\frac ab \times \frac cd = \frac{ac}{bd}
\end{equation}
\end{definition}

$(\mathbb{Q}, +, \times)$ is then a field:

\begin{enumerate}
\item $(\mathbb{Q}, +)$ is a commutative group;
\item $(\mathbb{Q}_0, \times )$ is a commutative group
\item $p (q+r) = pq + pr$
\end{enumerate}

\paragraph{Roots}

The square root of a number $a\geq 0$ is a $b \geq 0$ such that $b^2 = a$.

Show that in $\mathbb{Q}$ there is no $\sqrt{2}$:

\begin{proof}
By contradiction:\footnote{In the form: to show that $p \implies q$, we show that $\neg q \implies \neg p$, since
\begin{equation}
(p\implies q) \iff (\neg q \implies \neg p )
\end{equation}} if there were $a \in \mathbb{Z}$ and $b \in \mathbb{N}$ such that

\begin{equation}
\left( a\over b \right) ^2 = 2
\end{equation}

Suppose that $a, b$ have $\gcd(a, b) = 1$. Then $a^2 = 2 b^2$. So $a$ is even, and $a=2k$.

Then $4k^2 = 2b^2 \implies b=2n$: so $\gcd(a, b) \neq 1$
\end{proof}

With the same reasoning we can show that numbers such that $\sqrt{3}$ and $\sqrt[3]{2}$ are irrational.

\section{The set $\mathbb{R}$}

The set is given.
$\mathbb{R}$ is a completely ordered field:

\begin{enumerate}
\item $(\mathbb{R}, +)$ is a commutative group
\item $(\mathbb{R}_0, \times)$ is a commutative group
\item $a (b+c) = ab+ac$
\end{enumerate}

There also exists a relation called $\leq$, such that $\forall a, b, c \in \mathbb{R}$:

\begin{enumerate}
\item $a\leq a$
\item $a \leq b \wedge b \leq a \implies a = b$
\item $a \leq b \wedge b \leq c \implies a \leq c$
\item $a \leq b \vee b \leq a$ (\emph{completely} ordered)
\item $a \leq b \implies a+c \leq b+c$
\item $a \geq 0 \wedge b \geq 0 \implies ab \geq 0$ (of course, $a \geq 0$ means that $0 \leq a$) 
\end{enumerate}

$(\mathbb{Q}, +, \times) $ is also a completely ordered field?

Take the set of all the real numbers whose squares are greater or equal to 2: it has a minimum.

In $\mathbb{Q}$, it has no minimum.

Other statements: show that $a \leq 0 \wedge b \geq 0 \implies ab \leq 0$

\begin{proof}
Is $a \geq 0 \wedge -b \geq 0$? We first need to show that $a \geq 0 \iff -a \leq 0$: it suffices to add $-a$ to both sides. We also need to show that $a (-b) = -ab$: by an inverse application of the distributive property.
\end{proof}

\paragraph{Useful inequalities}

$\forall x \in \mathbb{R}: x^2 \geq 0$. Also, $\forall a, b \in \mathbb{R}: ab \geq (a^2 + b^2)/2$ (one of the inequalities between the means. From this, we can also show that between the rectangles of perimeter $p$, the square is the one with the largest area.

TO DO: show this for parallelograms and trapezes.

\paragraph{Integer part of a number} Given an $x \in \mathbb{R}$, we define its integer part $\lfloor x \rfloor = n \in \mathbb{Z}$ as the largest integer such that $n\leq x$.\footnote{This is also the case for negative numbers: so the integer part of a negative real number may have greater absolute value than the number itself.}

The fractionary part $\lbrace x \rbrace$ is defined as

\begin{equation}
\lbrace x \rbrace = x - \lfloor x \rfloor
\end{equation}

It is clear that $x-1 < \lfloor x \rfloor \leq x$ and that $0 \leq \lbrace x \rbrace  < 1$, and that $\lfloor x \rfloor =x \iff x \in \mathbb{Z}$

\section{Set notation}

The notation $[a; b]$ means $\left\lbrace x \in \mathbb{R}: a \leq x \leq b\right\rbrace$, and $(a; b) = ]a;b[$ means $\left\lbrace x \in \mathbb{R}: a < x < b\right\rbrace$. These are \emph{closed} and \emph{open} sets. We can combine the two types of brackets as we wish.

In the notation $(a; + \infty )$, the symbol $\infty$ is not a number.

\begin{definition}
We define the maximum of a set $E\subseteq \mathbb{R}$, denoted $\max E$ (and analogously $\min E$), as a number $M \in \mathbb{R}$ with the following properties:
\begin{enumerate}
\item $\forall x \in E: M \geq a$ ($M$ is an upper bound of $E$)
\item $M \in E$
\end{enumerate}
\end{definition}

A set is limited from above if it has an upper bound, and from below if it has a lower bound.
Open sets can be limited, but they do not have maximums and minimums: we can show this by taking the average between the first real number outside of the set and a number we suppose to be this maximum, getting to a contradiction.

\begin{definition}
We define the \emph{least upper bound} of a set $E$ as the minimum of the set of the upper bounds, and analogously for lower bounds. We can do this $\forall E \subseteq \mathbb{R}: E \neq \emptyset$, and we denote them as $\inf E$ and $\sup E$.
\end{definition}

If $E$ does not have an upper limit, we write $\sup E = + \infty$, but this is just notation. The same goes for $\inf E = -\infty$. It is also common to write $\sup \emptyset = - \infty$ and $\inf \emptyset = + \infty$, but defining them this way makes the fact $\inf E \leq \sup E$ not true.

$E$ has a maximum iff $\sup E \in E$. If $E \neq \emptyset \neq F$ and $E \subseteq F$, then $\sup E \leq \sup F$ and $\inf F \leq \inf E$.

\begin{theorem}[Archimedes' axiom]
Given $a, b \in \mathbb{R}$, where $a>0$ and $b>0$, $\exists n\in\mathbb{N}: na>b$.
\end{theorem}

\begin{proof}
We choose $n = \lfloor b/a \rfloor +1$.
\end{proof}

\paragraph{Axiom of continuity or completeness} Given $E \subseteq \mathbb{R}$, $E\neq\emptyset$, with at least an upper bound, there exists $\sup E \in \mathbb{R}$. The inverse is easily proven from this.

$\mathbb{R}$ (and sets which can be bijectively mapped to it, via a map that preserves addition and multiplication) is the only totally ordered set which verifies the axiom of continuity.

\begin{theorem}
$\sup E = M \iff \forall x \in M: x \leq M$ and $\forall \epsilon > 0: \exists z \in E : z>M-\epsilon$.
\end{theorem}

\begin{proof}
We will prove the leftward implication by contradiction. Suppose there exists an $M'<M$. Then $\forall \epsilon \in E: z \leq M'$. But define $\epsilon := (M-M')/2$:then ???
*Prove the rightward implication*
\end{proof}

\paragraph{Halved intervals}

\begin{theorem}
Take $a_k, b_k \in \mathbb{R}: \forall k \in \mathbb{N}: a_k < b_k$ and $\forall k \in \mathbb{N}: [a_{k+1}; b_{k+1}]$ is one of the halves of $[a_k; b_k]$. Then $\exists ! \lambda \in \mathbb{R}: \forall k \in \mathbb{N}: \lambda \in [a_k; b_k]$. We can write this as

\begin{equation}
\bigcap_{k=1}^{+\infty} [a_k; b_k] = \lbrace \lambda \rbrace
\end{equation}
\end{theorem}

\section{Topology}

We will focus on the topology of $\mathbb{R}$, sometimes generalizing to $\mathbb{R}^n$.

A \emph{metric space} is couple $(X, \text{d})$, where $\text{d}: X\times X \rightarrow \mathbb{R}$ is a \emph{distance} function, satisfying $\forall x, y, z \in X$:

\begin{enumerate}
\item $\text{d}(x, y) \geq 0$; $\text{d} (x, y) = 0 \iff x=y$;\label{positivity-distance}
\item $\text{d}(x, y) = \text{d}  (y, x)$;\label{symmetry-distance}
\item $\text{d}(x, y) \leq \text{d}(x, z) + \text{d}(y, z)$ (the triangular inequality).\label{triangular-distance}
\end{enumerate}

On $\mathbb{R}$, the distance function is usually $\text{d}(x, y) = \abs{x-y}$.\footnote{This is not our only option: something like $\text{d}(x, y) = \sqrt{\abs{x-y}}$ would work as well.}

This extends to $\mathbb{R}^n$:

\begin{equation}
\text{d}(x, y) = \sqrt{\sum_{i=0}^n (x_i - y_i)^2}
\end{equation}

This clearly satisfies conditions \ref{positivity-distance} and \ref{symmetry-distance}, but we have to prove condition \ref{triangular-distance}:

\begin{proof}
This follows from the subadditivity of the norm of vectors: $\mathbb{R}^n$ is a vector space, and we can interpret $\text{d}(x, y)$ as $\norm{\mathbf{x}-\mathbf{y}}$, so in the equation $\norm{\mathbf{x}+\mathbf{y}} \leq \norm{\mathbf{x}} + \norm{\mathbf{y}}$ we substitute $\mathbf{x}= u-z$ and $\mathbf{y}= z-v$.
\end{proof}

\subsection{Balls and openness}

Given the metric space $(X, d)$, with $x_0 \in X$ and $r>0, r\in \mathbb{R}$, we denote as $B(x_0, r)$ (or sometimes $B_r (x_0))$ the ``ball'':

\begin{equation}
B(x_0, r) = \lbrace x \in X: \text{d}(x, x_0) < r \rbrace
\end{equation}

For example, in $\mathbb{R}$ this looks like $(x_0 -r, x_0 + r)$.

\begin{definition}
Given the metric space $(X, d)$, and $E \subseteq X$, $x \in X$, we say that $x$ is internal to $E$ if $\exists r \in \mathbb{R}, r>0: B(x, r) \subseteq E$ (E intorno di X).
\end{definition}

\begin{definition}
Given the metric space $(X, d)$, and $E \subseteq X$, $x \in X$, we say that $x$ is external to $E$ if $\exists r\in \mathbb{R}, r>0: B(x, r) \cap E =\emptyset$. $x$ is thus internal to the complement of $E$.
\end{definition}

\begin{definition}
$x$ is a frontier (or boundary) point if it is neither internal nor external.
\end{definition}

We denote the set of internal points of $E$ as $\overset{\circ}{E}$ or $\text{int} E \subseteq E$, the boundary as $\partial E$, and the external points as $E^e$.

For example, in $\mathbb{R}^n$, if $E= B(x_0, r)$,  $\norm{x-x_0} < r \iff x \in \overset{\circ}{E}$ because of the triangular inequality: $B(x, (r- \norm{x-x_0})) \subseteq B(x_0, r)$. We can see that for any $y$ $\text{d}(y, x_0) \leq \text{d}(y, x) + \text{d} (x_0, x)$.

Also, if $\norm{x-x_0} > r$, $x \in E^e$, since we can construct $B(x, \norm{x-x_0} - r) \in E^C$.

If $\norm{x-x_0} = r$, $x \in \partial E$ since we can construct $B(x, p)$ for some $p>0$, and supposing  WLOG that $p<r$ we can show that:

\begin{align}
B(x, p) \diagdown E \neq \emptyset &\Leftarrow ( x \in E^C \implies \forall t>0 x \in B(x, t))\\
B(x, p) \cap E \neq \emptyset &
\end{align}

We can prove the last statement by defining

\begin{equation}
y:= x + \frac p2 \left( \frac{x-x_0}{r} \right)
\end{equation}

and showing that $\text{d}(y, x_0)<r$. 

\tableofcontents

\end{document}
