\documentclass[12pt,a4paper]{report}
\usepackage[utf8]{inputenc}
\usepackage{amsmath}
\newcommand{\norm}[1]{\left\lVert#1\right\rVert}
\usepackage{amsfonts}
\usepackage{comment}
\usepackage{nicefrac}
\usepackage[margin=3.3cm]{geometry}
\usepackage{amssymb}
\usepackage{accents}
\usepackage{amsthm}
\usepackage[pdftex, pdfborderstyle={/S/U/W 0}]{hyperref}
\numberwithin{equation}{section}
\usepackage{commath}
\usepackage{graphicx}
%\usepackage[extreme]{savetrees}
\usepackage{bm}
\usepackage{indentfirst}
\usepackage{nicefrac}
\setcounter{tocdepth}{4}
\usepackage{fnpct}
\usepackage{centernot}

\usepackage{cool}
\Style{DSymb={\mathrm d},DShorten=true,IntegrateDifferentialDSymb=\mathrm{d}}

\usepackage{microtype}
\usepackage{cleveref}

\theoremstyle{definition}
\newtheorem{definition}{Definition}[section]
\usepackage{mathtools}
 
\theoremstyle{remark}
\newtheorem*{remark}{Remark}

\newtheorem{theorem}{Theorem}[section]
\newtheorem{corollary}{Corollary}[theorem]
\newtheorem{lemma}[theorem]{Lemma}

\usepackage{url}
\newcommand*{\defeq}{\stackrel{\text{def}}{=}}
\author{Jacopo Tissino}
\title{Notes on Calculus I}

\begin{document}

\maketitle

\chapter{Naïve set theory}

\section{Basic sets}

\begin{equation}
\mathbb{N} \subseteq \mathbb{Z} \subseteq \mathbb{Q} \subseteq \mathbb{R}, \qquad 0 \not \in \mathbb{N}
\end{equation}

\begin{equation}
\mathbb{R}^+ := \lbrace x \in \mathbb{R} : x>0 \rbrace
\end{equation}

\paragraph{Remarks}

\begin{itemize}
\item ``$\in$'' is for elements belonging to sets, ``$\subseteq$'' is for subsets
\item $\lbrace x \rbrace \neq x$: the first is a set with $x$ as its only element, and is called a ``singlet''
\item $\subsetneq$  means ``is a subset of, but not equal to''
\item the elements of $\mathcal{P}(A)$ are precisely all the subsets of $A$
\item $\sharp A$ is the cardinality of $A$
\item $\sharp \mathcal{P} (A) = 2^{\sharp A}$
\end{itemize}

The naïve definitions of $A \cup B$, $A \cap B$, $A \smallsetminus B$ are given.

\paragraph{Properties}

\begin{itemize}
\item $A = (A \cap B ) \cup (A \smallsetminus B)$
\item $(A \cap B) \cap (A \smallsetminus B) = \emptyset$
\item $C \cap (A \cup B) = (C \cap A) \cup (C \cap B)$
\item $C \cup (A \cap B) = (C \cup A) \cap (C \cup B)$
\end{itemize}

\paragraph{Complement}

\begin{definition}
With respect to a ``universe'' set $U$, we define the complement of $A$ as $U \smallsetminus A$, denoted $A^C$.
\end{definition}

The following hold:

\begin{itemize}
\item $(A \cup B)^C = A^C \cap B^C$
\item $(A \cap B)^C = A^C \cup B^C$
\end{itemize}

\paragraph{Cartesian product}

\begin{definition}
An \emph{ordered pair} is a set of the form $\lbrace \lbrace x \rbrace , \lbrace x, y \rbrace \rbrace$, denoted $(x, y)$ (where order matters).
\end{definition}

\begin{definition}
We define the \emph{cartesian product} $A\times B$ of two sets $A$ and $B$ as:

\begin{equation}
A\times B := \left\lbrace (a, b) : a \in A, b \in B \right\rbrace
\end{equation}
\end{definition}

\section{Propositional logic}

\paragraph{Implication}
\begin{definition}
\begin{equation}
p\implies q \iff (\neg p ) \vee q
\end{equation}
\end{definition}

\paragraph{Double implication}
\begin{definition}
\begin{equation}
(p \iff q) \iff (p \implies q \wedge q \implies p )
\end{equation}
\end{definition}

\paragraph{Quantifiers}

$P(x)$ is a \emph{predicate}. We say that $\forall x : P(x)$ if $P(x)$ is true independently of $x$, and that

\begin{equation}
\exists x : P(x) \iff \neg ( \forall x : \neg P(x) )
\end{equation}

\tableofcontents

\end{document}
