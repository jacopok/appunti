\documentclass[12pt,a4paper]{article}
\usepackage{textcomp}
%% Language and font encodings
\usepackage[utf8x]{inputenc}
\usepackage[T1]{fontenc}
\usepackage[italian]{babel}
\usepackage{multirow}
\usepackage{float}
\usepackage[caption = false]{subfig}
\usepackage{longtable}
\usepackage{listings}
\usepackage{mathtools}
\usepackage{comment}
\DeclarePairedDelimiter{\abs}{\lvert}{\rvert}

\lstset{%
  basicstyle=\ttfamily,
  breaklines=true,
  columns=fullflexible
}


%% Sets page size and margins
\usepackage[a4paper,top=3cm,bottom=2cm,left=3cm,right=3cm,marginparwidth=1.75cm]{geometry}

%% Useful packages
\usepackage{amsmath}
\usepackage{graphicx}
\usepackage[colorinlistoftodos]{todonotes}
\usepackage[colorlinks=true, allcolors=blue]{hyperref}
\usepackage{siunitx}
\sisetup{separate-uncertainty=true}

\title{***}
\author{Alberto Facheris, Tommaso Michelutti, Jacopo Tissino}

\begin{document}
\maketitle

\section{Scopo dell'esperimento} %check

\section{Apparato sperimentale} % Alberto

\section{Metodo}

\section{Analisi dati}



\section{Discussione} % Jacopo

\section{Conclusione} % Jacopo

\appendix

\clearpage


\section{Formule utilizzate} % Jacopo

Per un campione $x = \lbrace x_i \rbrace _{i = 1 \dots N}$, utilizziamo le seguenti formule:

\begin{equation}
\bar{x} = \frac{1}{N} \sum_{i=1}^N x_i
\end{equation}

\begin{equation}
\sigma_x = \sqrt{\frac{\displaystyle \sum_{i=1}^N (x_i - \bar{x})^2}{N-1}}
\end{equation}

\begin{equation}
\sigma_{\bar{x}} = \frac{\sigma_x}{\sqrt{N}}
\end{equation}

Per un campione di coppie $\lbrace (x_i, y_i) \rbrace _{i = 1 \dots N}$ possiamo trovare la retta di regressione $y = a + b x$ con le seguenti formule:

\begin{equation}
\Delta = 
N \sum_{i=1}^N x_i^2
-
\left(
\sum_{i=1}^N x_i
\right) ^2
\end{equation}

\begin{equation}
a = \Delta ^{-1} \left(
\sum_{i=1}^N x_i^2
\right)
\left(
\sum_{i=1}^N y_i
\right)
- \Delta ^{-1}
\left(
\sum_{i=1}^N x_i
\right)
\left(
\sum_{i=1}^N x_i y_i
\right)
\end{equation}

\begin{equation}
b = N\Delta ^{-1}
\left(
\sum_{i=1}^N x_i y_i
\right)
- \Delta^{-1}
\left(
\sum_{i=1}^N x_i
\right)
\left(
\sum_{i=1}^N y_i
\right)
\end{equation}

L'errore a posteriori su $y$ $\sigma _y$ può essere calcolato come segue:

\begin{equation}
\sigma_y =
\sqrt{\frac{ \displaystyle
\sum_{i=1}^N \left( 
a + b x_i - y_i
\right)^2
}{N-2}}
\end{equation}

Gli errori su $a$ e $b$ possono essere calcolati come segue, utilizzando come $\sigma_y$ l'errore misurato invece di quello a posteriori se possibile:

\begin{equation}
\sigma_a = \sigma_y \sqrt{\Delta ^{-1} \displaystyle \sum_{i=1}^N x_i^2 }
\end{equation}

\begin{equation}
\sigma_b = \sigma_y \sqrt{N\Delta^{-1}}
\end{equation}

Per calcolare la propagazione dell'errore in una misura indiretta esprimibile come $f(x_1, x_2, \dots, x_n)$, con le $x_i$ indipendenti, utilizziamo la formula

\begin{equation}
\sigma_f = \sqrt{\sum_{i=1}^N \left( \sigma_{x_i} \frac{\partial f}{\partial x_i} \right)^2 }
\end{equation}

Per calcolare la media pesata $\langle x \rangle$ di una quantità $x$ di cui abbiamo diverse stime $\lbrace x_i \rbrace _{i=1 \dots N}$ con i relativi errori $\sigma_i$ utilizziamo la seguente formula:

\begin{equation}
\langle x \rangle = \frac{
\displaystyle \sum_{i=1}^N \frac{x_i}{\sigma_i^2}
}{ \displaystyle
\sum_{i=1}^N \sigma_i ^{-2}
}
\end{equation}

con il relativo errore della media

\begin{equation}
\sigma_{\langle x \rangle} = \left(
\sum_{i=1}^N \sigma_i ^{-2}
\right) ^{-\frac 12}
\end{equation}

Per calcolare il coefficiente di correlazione $r$ fra due set $(x_i)$ e $(y_i)$ utilizziamo la seguente formula:

\begin{equation}
r = \frac{\text{cov} (x, y)}{\sigma_x \sigma_y}
\end{equation}

e definiamo

\begin{equation}
T_n = \frac{r \sqrt{n-2}}{\sqrt{1-r^2}}
\end{equation}

\newpage

\section{Grafici e tabelle complete dei dati} % Tommaso

\section{Dati rilevati}

\clearpage

\section{Programma utilizzato per l'analisi dati} % Tommaso

%\begin{lstlisting}[breaklines]
%\end{lstlisting}

\end{document}