\documentclass[12pt,a4paper]{report}
\usepackage[utf8]{inputenc}
\usepackage{amsmath}
\newcommand{\norm}[1]{\left\lVert#1\right\rVert}
\usepackage{amsfonts}
\usepackage{comment}
\usepackage{nicefrac}
\usepackage[margin=3.3cm]{geometry}
\usepackage{amssymb}
\usepackage{accents}
\usepackage{amsthm}
\usepackage[pdftex, pdfborderstyle={/S/U/W 0}]{hyperref}
\numberwithin{equation}{section}
\usepackage{commath}
\usepackage{graphicx}
%\usepackage[extreme]{savetrees}
\usepackage{bm}
\usepackage{indentfirst}
\usepackage{nicefrac}
\setcounter{tocdepth}{4}
\usepackage{fnpct}
\usepackage{centernot}

\usepackage{cool}
\Style{DSymb={\mathrm d},DShorten=true,IntegrateDifferentialDSymb=\mathrm{d}}

\usepackage{microtype}
\usepackage{cleveref}

\theoremstyle{definition}
\newtheorem{definition}{Definition}[section]
\usepackage{mathtools}
\usepackage{commath}
 
\theoremstyle{remark}
\newtheorem*{remark}{Remark}

\newtheorem{theorem}{Theorem}[section]
\newtheorem{corollary}{Corollary}[theorem]
\newtheorem{lemma}[theorem]{Lemma}

\usepackage{url}
\newcommand*{\defeq}{\stackrel{\text{def}}{=}}
\author{Jacopo Tissino}
\title{Notes on Statistics}

\begin{document}

\maketitle

\section{Distribuzione gaussiana}

È una distribuzione degli scarti $z_i = x_i - x_{\text{ref}}$, ricavata dalle seguenti premesse:

\begin{enumerate}
\item Se ci sono solo errori casuali, la densità di probabilità degli scarti deve essere simmetrica rispetto allo zero;
\item Gli scarti grandi sono poco probabili: $\lim_{z\rightarrow \infty} = 0$;
\item Gli scarti piccoli sono molto probabili: $\lim_{z\rightarrow 0} = \max{f(z)}$.
\item $f(z) = \frac{\dif p}{\dif z}$.
\end{enumerate}

Gauss ha dimostrato che deve avere l'espressione:

\begin{equation}
f(z) = k e^{-h^2 z^2}
\end{equation}

dove $k$ è una costante di normalizzazione e $h$ è un parametro detto \emph{modulo di precisione}.

Per trovare il valore di $k$ poniamo

\begin{equation}
k \int_{\mathbb{R}} e^{-h^2 z^2} \, \dif z = 1
\end{equation}

che ci dà $k = h /\sqrt{\pi}$. Sembra che $h$ sia correlato a $\sigma$; vediamo che 

\begin{equation}
E(z) = \int_{\mathbb{R}} z \frac{h}{\sqrt{\pi}} e^{-h^2 z^2} \, \dif z = 0
\end{equation}

per simmetria, e dunque $E(z) = x^* = 0$. 
Questo semplifica il calcolo della varianza:

\begin{equation}
\var (
\end{equation}

\tableofcontents


\end{document}
